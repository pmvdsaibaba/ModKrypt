\begin{itemize}
\item [(a)]
	\textbf{To show:} Prove that regular CPA security implies \(\lambda\)-CPA security.\\
	We do this by a reduction. We assume there is an efficient adversary \(\mathcal{A}\) against the \(\lambda\)-CPA-security of \(\Pi\). From this we construct our adversary \(\mathcal{B}\) against the CPA-security of \(\Pi\).
\item [(b)]
	\textbf{To show:} Prove that \(\lambda\)-CPA security implies normal CPA security\\
	We do this by a reduction. We assume there is an efficient adversary \(\mathcal{A}\) against the CPA-security of \(\Pi\). From this we construct our adversary \(\mathcal{B}\) against the \(\lambda\)-CPA-security of \(\Pi\) which invokes \(\mathcal{A}\). \(\mathcal{B}\) has to provide an encryption oracle for \(\mathcal{A}\) To do this he forwards any message \(m\) \(\mathcal{A}\) sends to his oracle to his own oracle and recieves the ciphertextvector \(C_b\). He then forwards only the first ciphertext \(c_1\) to \(\mathcal{A}\). \\
	\(\mathcal{A}\) eventually outputs two messages \((\widetilde{m_0}, \widetilde{m_1})\), which \(\mathcal{B}\) forwards to his challenger. From the recieved ciphertextvector \(C_b\) he again forwards only the first ciphertext to \(\mathcal{A}\). Then \(\mathcal{B}\) outputs the same bit \(b\) like \(\mathcal{A}\) does. 
\end{itemize}

