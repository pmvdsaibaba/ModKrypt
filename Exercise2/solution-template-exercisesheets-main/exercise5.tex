\begin{itemize}
    \item[(a)]
        For the case when length of message, $l_m$, is greater than  length of key,  $l_k$: $l_m > l_k$:\\\\
        Then some bits of the key or the whole key is used more than once for encryption. It is like using
        one time pad multiple times with same key. \\
        
        As mention in page 36 of \cite{texbook}, \textit{"the one-time pad scheme — as the name indicates — is
        only 'secure' if used once (with the same key). ................ it is easy to see
        informally that encrypting more than one message leaks a lot of information.
        In particular, say two messages are encrypted using the same key k.
        and thus learn something about the exclusive-or of the two messages. While
        this may not seem very significant, it is enough to rule out any claims of perfect
        secrecy when encrypting two messages. Furthermore, if the messages correspond to
         English language text, then given the exclusive-or of sufficiently
        many message pairs it is possible to perform frequency analysis .... and recover the messages themselves."}\\

        Hence this scheme is not perfectly secure for $l_m > l_k$.\\

    \item[(b)]
        This encryption scheme can be made secure by atleast restricing the length of key as close as length of message.
        such that $ (l_m - l_k) $ is almost close to zero or $ (l_k / l_m) $ is almost approximated to 1. \\
    
        or contacting different keys in such a way $l_k > l_m$.\\\\ 
\end{itemize}

\begin{thebibliography}{9} 
    \bibitem{texbook}
    Katz, J., \& Lindell, Y. (2014). Introduction to Modern Cryptography (2nd ed.). Chapman and Hall/CRC. https://doi.org/10.1201/b17668

\end{thebibliography}


