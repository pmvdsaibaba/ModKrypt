\begin{equation*}
 	\begin{array}{ll}
 		L_0 & R_0\\
 		L_1= R_0 & R_1 = L_0 \xor f_1(R_0)\\
 		L_2= R_1 = L_0 \xor f_1(R_0) & R_2 = L_1 \xor f_2(R_1) = R_0 \xor f_2( L_0 \xor f_1(R_0))\\
 		L_3= R_2 = R_0 \xor f_2( L_0 \xor f_1(R_0)) & R_3 = L_2 \xor f_3(R_2) = L_0 \xor f_1(R_0) \xor f_3(R_0 \xor f_2( L_0 \xor f_1(R_0)))\\
 	\end{array}
\end{equation*}
\begin{equation*}
	\begin{array}{l}
		L_4= R_3 =  L_0 \xor f_1(R_0) \xor f_3(R_0 \xor f_2( L_0 \xor f_1(R_0))) \\
		R_4 = R_0 \xor f_2( L_0 \xor f_1(R_0)) \xor f_4(L_0 \xor f_1(R_0) \xor f_3(R_0 \xor f_2( L_0 \xor f_1(R_0)))) \\
	\end{array}
\end{equation*}

Inversion: 
\begin{equation*}
	\begin{array}{ll}
		L_3 & R_3\\
		L_2= R_3 \xor f_3(R_2) = R_3 \xor f_3(L_3) & R_2 = L_3 \\
		L_1= R_2 \xor f_2(R_1) = L_3 \xor f_2(R_3 \xor f_3(L_3)) & R_1 = L_2 = R_3 \xor f_3(L_3)\\
		L_0 = R_1 \xor f_1(R_0) = R_3 \xor f_3(L_3) \xor f_1(L_3 \xor f_2(R_3 \xor f_3(L_3))) & R_0 = L_1 = L_3 \xor f_2(R_3 \xor f_3(L_3))\\
	\end{array}
\end{equation*}

\begin{itemize}
	\item [1.]
		\textbf{To show:} Prove that a two-round Feistel network using pseudorandom round functions is not a pseudorandom permutation.\\
		The adversary first queries a random string \(L_0 \abs{} R_0\) to the oracle. After two rounds he gets \(L_2 \abs{} R_2 = L_0 \xor f_1(R_0) \abs{} R_0 \xor f_2( L_0 \xor f_1(R_0))\) back, if the oracle answers with the PRP. Then he queries a second string \(L_0^* \abs{} R_0\), where \(L_0^*\) is \(L_0\) with the first bit flipped. 
		He then gets \(L_2^* \abs{} R_2^* = L_0^* \xor f_1(R_0) \abs{} R_0 \xor f_2( L_0^* \xor f_1(R_0))\) back, if the oracle answers with the PRP. It is easy to check whether \(L_2\) and \(L_2^*\) differ only in the first bit. This is the case when the oracle answers with the two-round Feistel network. If not, the adversary knows, that the oracle answers with a random string. \\
		So he can distinguish between the two-round Feistel network and a random permutation with non-negligible probability, so a two-round Feistel network is not a pseudorandom permutation. 
	\item [2.]
		\textbf{To show:} Prove that a three-round Feistel network using pseudorandom round functions is not a strongly pseudorandom permutation.
\end{itemize}