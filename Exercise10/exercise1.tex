\begin{equation*}
 	\begin{array}{ll}
 		L_0 & R_0\\
 		L_1= R_0 & R_1 = L_0 \xor f_1(R_0)\\
 		L_2= R_1 = L_0 \xor f_1(R_0) & R_2 = L_1 \xor f_2(R_1) = R_0 \xor f_2( L_0 \xor f_1(R_0))\\
 		L_3= R_2 = R_0 \xor f_2( L_0 \xor f_1(R_0)) & R_3 = L_2 \xor f_3(R_2) = L_0 \xor f_1(R_0) \xor f_3(R_0 \xor f_2( L_0 \xor f_1(R_0)))\\
 	\end{array}
\end{equation*}
\begin{equation*}
	\begin{array}{l}
		L_4= R_3 =  L_0 \xor f_1(R_0) \xor f_3(R_0 \xor f_2( L_0 \xor f_1(R_0))) \\
		R_4 = R_0 \xor f_2( L_0 \xor f_1(R_0)) \xor f_4(L_0 \xor f_1(R_0) \xor f_3(R_0 \xor f_2( L_0 \xor f_1(R_0)))) \\
	\end{array}
\end{equation*}

Inversion: 
\begin{equation*}
	\begin{array}{ll}
		L_3 & R_3\\
		L_2= R_3 \xor f_3(R_2) = R_3 \xor f_3(L_3) & R_2 = L_3 \\
		L_1= R_2 \xor f_2(R_1) = L_3 \xor f_2(R_3 \xor f_3(L_3)) & R_1 = L_2 = R_3 \xor f_3(L_3)\\
		L_0 = R_1 \xor f_1(R_0) = R_3 \xor f_3(L_3) \xor f_1(L_3 \xor f_2(R_3 \xor f_3(L_3))) & R_0 = L_1 = L_3 \xor f_2(R_3 \xor f_3(L_3))\\
	\end{array}
\end{equation*}

\begin{itemize}
	\item [1.]
		\textbf{To show:} Prove that a two-round Feistel network using pseudorandom round functions is not a pseudorandom permutation.\\
		The adversary first queries a random string \(L_0 \abs{} R_0\) to the oracle. After two rounds he gets \(L_2 \abs{} R_2 = L_0 \xor f_1(R_0) \abs{} R_0 \xor f_2( L_0 \xor f_1(R_0))\) back, if the oracle answers with the PRP. Then he queries a second string \(L_0^* \abs{} R_0\), where \(L_0^*\) is \(L_0\) with the first bit flipped. 
		He then gets \(L_2^* \abs{} R_2^* = L_0^* \xor f_1(R_0) \abs{} R_0 \xor f_2( L_0^* \xor f_1(R_0))\) back, if the oracle answers with the PRP. It is easy to check whether \(L_2\) and \(L_2^*\) differ only in the first bit. This is the case when the oracle answers with the two-round Feistel network. If not, the adversary knows, that the oracle answers with a random string. \\
		So he can distinguish between the two-round Feistel network and a random permutation with non-negligible probability, so a two-round Feistel network is not a pseudorandom permutation. 
	\item [2.]
		\textbf{To show:} Prove that a three-round Feistel network using pseudorandom round functions is not a strongly pseudorandom permutation.

		
		\textbf{Note:} Inversion formula stated in lecture note is wrong.
		
		position of L3 and R3 should be as $ Data = (R3||L3)$ 

			% $L_2= R_3 \xor f_3(R_2) = R_3 \xor f_3(L_3)  R_2 = L_3 $\\
			% $L_1= R_2 \xor f_2(R_1) = L_3 \xor f_2(R_3 \xor f_3(L_3)) & R_1 = L_2 = R_3 \xor f_3(L_3)$\\


			% $L_0 = R_3 \xor f_3(L_3) \xor f_1(L_3 \xor f_2(R_3 \xor f_3(L_3)))$
			

			% $R_0 = L_3 \xor f_2(R_3 \xor f_3(L_3))$


			In lecture note, the position of L3 and R3 is opposite i.e., $ Data = (L3||R3)$ This convention is wrong.


			So for avoiding confusion, if  $ Data = (L3||R3)$, then 

			$L_0 = R_3 \xor f_2(L_3 \xor f_3(R_3))$

			$R_0 = L_3 \xor f_3(R_3) \xor f_1(R_3 \xor f_2(L_3 \xor f_3(R_3)))$

			
			Now coming to proof:


			Let us generalize this notation as $D_{k_1,k_2,k_3} \left (L||R\right ) = \left (x||y\right ) $ as the inversion formula for third order.			
			And $E_{k_1,k_2,k_3} \left (L||R\right ) = \left (x||y\right ) $ as normal feistel network formula.


			Here is a Adversary that disingushes with high probability

			\begin{itemize}
				\item Query the decryption oracle with two strings of zero bits: $(a||b) \leftarrow D(0||0)$
				\item Query the encryption oracle:  $(c||d) \leftarrow E(0||a)$ 
				\item Query the decryption oracle again: $(e||f) \leftarrow D(b \xor d ||c)$ 
				\item If $ e= c\xor a$ then return 1, else return 0
			\end{itemize}
			
			Explanation:
			\begin{itemize}
				\item Query the decryption oracle with two strings of zero bits: $(a||b) \leftarrow D(0||0)$\\
				By using expansion of $D$ \\
				$ a =f_2(f_3(0)) $\\
				$ b= f_3(0) \xor f_1(a)$
				\item Query the encryption oracle:  $(c||d) \leftarrow E(0||a)$ 
				By using expansion of $E$ \\
				$ c = a\xor f_2(f_1(a)) $\\
				$ d = f_1(a) \xor f_3(c)$


				now on calculating $b\xor d = f_3(0) \xor f_3(c)$ we see that $f_1(a)$ cancels out.
				\item Query the decryption oracle again: $(e||f) \leftarrow D(b \xor d ||c)$ 
				Now again by using the expansion of $D$ we get 

				$e = c \xor a$
				\item Hence the adversary can check if $ e == c\xor a$ then differentiate if the output is from feistel network or random.
			\end{itemize}
			Reference: https://crypto.stackexchange.com/questions/32974/example-of-a-prp-that-is-not-a-strong-prp



\end{itemize}