\textbf{To show:} If an algorithm \(A\) exists that can solve the CDH problem in poly time with probability \(\epsilon\), then an algorithm \(B\) exists that can decide the DDH problem in poly time with probability \(\epsilon - \frac{1}{q}\). \\ \\
We show this by reduction. We assume there is an efficient adversary \(\mathcal{A}\) against the CDH problem that manages to compute \(g^z\) from \(g^x\) and \(g^y\) with \(z = x \cdot y\) given a group \(G\) of order \(q\) with generator \(g\).\\
We construct \(\mathcal{B}\) against the DDH problem which invokes \(\mathcal{A}\). \(\mathcal{B}\) recieves a group \(\mathcal{G}\) of order \(q\) with generator \(g\) as well as \(g^x\), \(g^y\) and \(g^{z_b}\) with \(z_b\) being \(x \cdot y\) for \(b = 0\) or a random element of the group for \(b = 1\). \(\mathcal{B}\) forwards the group \(G\) of order \(q\) with generator \(g\) as well as \(g^x\), \(g^y\) to \(\mathcal{A}\). \(\mathcal{A}\) outputs \(g^z\), which \(\mathcal{B}\) compares to its recieved \(g^{z_b}\).\\
If \(g^z = g^{z_b}\) \(\mathcal{B}\) outputs \(b' = 1\) otherwise \(b' = 0\).  If \(b = 0\) the probablity of \(g^z = g^{z_b}\) \(\leftrightarrow\) \(g^z = g^{xy}\) is \(\epsilon\) since the CDH problem can be solved with probability \(\epsilon\).  If \(b = 1\) \(z_b\) is a random element of the the group \(G\) with \(z_b \in\{0, q-1\}\). For \(z_b\) to be coincidentally equal to \(z\) is \(\frac{1}{q}\) since there is only one out of \(q\) numbers of the group can be equal to \(z\). \\
\(\mathcal{B}\) invokes \(\mathcal{A}\) and \(\mathcal{A}\) is efficient. \(\mathcal{B}\) only forwards parts of its input and compares \(\mathcal{A}\)'s output and is therefor efficient as well.\\
\(\mathcal{B}\) simulates the DDH problem perfectly. The success probability is \(\epsilon\) for \(b = 0\) and \(\frac{1}{q}\) for \(b = 1\) making \(Pr[\mathcal{B}(G, q, g, g^x, g^y, g^{z_0})] - Pr[\mathcal{B}(G, q, g, g^x, g^y, g^{z_1})] = \epsilon - \frac{1}{q}\).\\