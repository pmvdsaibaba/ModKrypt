\begin{itemize}
	\item [(a)]
	\textbf{To show:} A group contains exactly one identity element. That is, if \(e\) and \(1\) are both identities
	in a group \(\mathds{G}\), then \(1 = e\).\\
	\(e\) is an identity: \(e \circ g = g = g \circ e \hspace{0.5cm} \forall g \in \mathds{G}\)\\
	\(1\) is an identity: \(1 \circ g = g = g \circ 1 \hspace{0.5cm} \forall g \in \mathds{G}\)
	
	\begin{equation*}
		\begin{array}{rll}
			g &=g &\\
			e \circ g &= 1 \circ g & \vert \circ g^{-1}\\
			e \circ e &= 1 \circ e & \\
			e &= 1 &			
		\end{array}
	\end{equation*}

	\item [(b)]
	\textbf{To show:} If an element \(l\) is a left identity element of \(\mathds{G}\), which means that \(l \circ g = g\) for all \(g \in \mathds{G}\), then it is the identity element.\\
	\(l\) is an left identity: \(l \circ g = g \hspace{0.5cm} \forall l \in \mathds{G}\)\\
	\begin{equation*}
		\begin{array}{rll}
			l \circ g &=g & \vert \circ g^{-1}\\
			l \circ e &= g \circ g^{-1} & \vert  g \circ\\
			g \circ l &= g = l \circ g & \\		
		\end{array}
	\end{equation*}

	\item [(c)]
	\textbf{To show:} The inverse of \(g \circ h\) is \(h^{-1} \circ g^{-1}\).
	\begin{equation*}
		(g \circ h) \circ (h^{-1} \circ g^{-1}) = g \circ (h \circ h^{-1}) \circ g^{-1} = g \circ e \circ g^{-1} = g  \circ g^{-1} = e			
	\end{equation*}
	\item [(d)]
	\textbf{To show:} \(g^n\) denotes \(e \circ g \circ g \circ ... \circ g\), with \(g\) appearing \(n\) times (for \(n \ge 0\)). Show that \((g^{-1})^n\) is the inverse of \(g^n\) for all \(n \ge 0\). (For this reason we define \(g^{-n} := (g^{-1})^n\).)
	\begin{equation*}
		\begin{array}{l}
		g^n \circ (g^{-1})^n = (e \circ g \circ g \circ ... \circ g \circ g) \circ (e \circ g^{-1} \circ g^{-1} \circ ... \circ g^{-1} \circ g^{-1}) \\
		= (g \circ g \circ ... \circ g \circ g) \circ (g^{-1} \circ g^{-1} \circ ... \circ g^{-1} \circ g^{-1})
		= (g \circ (g \circ ... \circ  (g\circ (g \circ g^{-1}) \circ g^{-1}) \circ ... \circ g^{-1})\circ g^{-1})\\
		= (g \circ (g \circ ... \circ  (g\circ e \circ g^{-1}) \circ ... \circ g^{-1})\circ g^{-1})
		= (g \circ (g \circ ... \circ  (g \circ g^{-1}) \circ ... \circ g^{-1})\circ g^{-1})
		= e
		\end{array}		
	\end{equation*}

	\item [(e)]
	\textbf{To show:} Show that if \(\mathds{G}\) is finite and \(g \in \mathds{G}\), then there exists an \(n > 0\) such that \(g^n = e\). (The smallest such \(n\) is called the order of \(g\).)

	As $\mathbb{G}$ is finite the powers $g^1$,$g^2$,$g^3$,..... are not all distinct.
	Let's assume there exists $x<y$ such that $g^x = g^y$.

	That implies $g^{y-x} = e$ and $y-x >0 $. To satisy this and our assumption, there should a positive number $n$ such that
	$g^n= e$.
\end{itemize} 