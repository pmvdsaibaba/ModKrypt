\textbf{Task:} Prove that indistinguishability of multiple encryptions in the presence of an eavesdropper does
not imply indistinguishability of encryptions under a chosen plaintext attack.\\
\\
We show this by a proof by contradiction. Let's assume that an arbitrary encryption scheme \(\Pi = (Gen, Enc, Dec)\) exists, that is EAV-Mult secure.\\
\(\Pi' = (Gen', Enc', Dec')\) is constructed as follows:\\
\(Gen' = Gen\)\\
\(Enc'\) and \(Dec'\) are constructed similar to the Chained Cipher Block Chaining: \\
\(\underline{Enc'(k, m)}:\)\\
\(c \leftarrow Enc(k, m)\)\\
\(if\ m = r:\), r is choosen uniformly at random, but is fixed for all encryptions\\
\(\indent return\ r,k,c\)\\
\(else:\)\\
\(\indent return\ r,0,c\)\\

%\(\underline{Enc'(k, m)}:\)\\
%\(IV \leftarrow \{0,1\}^{\lambda}\)\\
%\(c_0 := IV\), if invoked for the first time\\
%\(c_0 := c_n^{last\_invoke\_from\_Enc'}\)\\
%\(for\ i = 1, ..., l\ do \)\\
%\(\indent c_i = Enc(k, c_{i-1} \xor m)\)\\
%\(return\ (c_0, ..., c_n)\)\\

\noindent \(\underline{Dec'(k, c)}:\)\\
\(parse(\perp, \perp, c)\)\\
\(m \leftarrow  Dec(k,c)\)\\
\(return\)\\
\\
Proof, that \(\Pi'\) remains EAV-mult secure:\\
Proof by contradiction: We assume there is an adversary \(\mathcal{A}\) who can break \(\Pi'\). We then construct an adversary \(\mathcal{B}\) against \(\Pi\) as follows: The adversary \(\mathcal{B}\) has access to the encryption function \(Enc'\). To use the adversary \(\mathcal{A}\) against the security of \(\Pi'\), \(\mathcal{B}\) has to provide the interface for \(\mathcal{A}\). He does this by simulating the encryption for \(\mathcal{A}\). \(\mathcal{A}\) outputs two message vector  \((m_0^1, m_0^2,..., m_0^t)\) and \((m_1^1, m_1^2,..., m_1^t)\) with \(\vert m_0^i \vert = \vert m_1^i \vert \forall i\). The adversary \(\mathcal{B}\) than samples a bit \(b \leftarrow\$ \{0,1\}\) and invokes his own encryption function \(Enc(k,m)\) to compute \(tup^i = \left\{
\begin{array}{ll}
	(r,k, Enc(k,m_b^i)) & if\ m_b^i = r\\
	(r, 0, Enc(k,m_b^i)) & else\\
\end{array}
\right.	\)\\
Then he forwards \(tup^1, tup^2, ... tup^t\) to \(\mathcal{A}\). Eventually \(\mathcal{A}\) outputs a bit \(b'\). The adversary \(\mathcal{B}\) outputs the same bit \(b'\).
\\
Describe a successful adversary against the CPA security of \(\Pi'\):\\
For the first call to the enncryption oracle, the adversary can choose the message randomly. It is very unlikly that he guesses the message so that \(m = r\). In this case, he would learn \(k\) immediatly. If the adversary doesn't guess the correct message he learns \(r\). With this information he can choose the next message correct with \(m = r\) and then learn \(k\). \\
Because he knows all about the encryption except \(k\) yet, he now knows the complete encryption scheme and can distinguish for any two messages sent to the challenger, from which message he recieves the ciphertext. So the adversary can win the game with a probability of 1 which is greater than a negligible function.