Let adversary $A$ can choose can choose messages $x_1$ and $x_2$. Then corresponding keyed functions adversary
\begin{center}
    $ F_k(x_1) := G_n(k) \xor x_1 $ \\
    $ F_k(x_2) := G_n(k) \xor x_2 $ \\
\end{center}

$A$ computes $ F_k(x_1) \xor F_k(x_2) $ which will be $ x_1 \xor x_2 $ \\

With this A can easily distingush from a uniformly selected function $f$ by checking 

\begin{center}
    $ (F_k(x_1) \xor F_k(x_2) ) \xor x_1 => x_1 \xor x_2 \xor x_1  = x_1$
\end{center}

or 
\begin{center}
    $ (F_k(x_1) \xor F_k(x_2) ) \xor x_2 => x_1 \xor x_2 \xor x_2  = x_1$
\end{center}

Hence this will not be a PRF as this keyed function can be distinguished from a uniformaly selected function.

% Let $G$ be a pseudorandom number generator with output of length 2n. We split the output in two 
% parts as 
% \begin{center}
%     $ G(x) := G^0(x) || G^1(x) \quad with \quad G^b(x) \in \{0,1\}^n $
% \end{center}


% So $ |G(s)| = 2n $.\\


% Given the construction $ G_n(s) $, denoted as the n-bit prefix of $ G(s)$, i.e.,
% \begin{center}
%     $ G_n(s) =   G^0(x) || 0^n $
% \end{center}

% And given the keyed function
% \begin{center}
%     $ F_k(x) =   G_n(k) \xor x $
% \end{center}

% so
% \begin{center}
%     $ F_k(x) =   G^0(x) || 0^n \xor x $
% \end{center}


% This shows us that last $n$ bits in this case is same as the input x. Hence this function is not a PRF as this function can 
% be distinguished from a uniformly selected function $f$ by checking if the last $n$ bits of
% the output is same as the input $x$.
