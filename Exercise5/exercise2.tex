Assume n = 3 \\
The Adversary  \(\mathcal{A}\) can choose the two messages m\(_{0}\)  = 001 001 and  m\(_{1}\)  = 110 011 \\
The challenger encrypted one of the messages. \\
Since f is a pseudorandom permutation, f is deterministic and has a reverse operation. \\
\\
If m\(_{0}\)  is encrypted: \\
\>Enc(m\(_{0}\)) = Enc(m\(_{0}^{0}\) \(\vert \vert\) m\(_{0}^{1}\)) = Enc(001 001) = f(m\(_{0}^{0}\)) \(\vert \vert\) f(m\(_{0}^{1}\)) =
f(001) \(\vert \vert\) f(001) = c\(^{0}\) \(\vert \vert\) c\(^{1}\) \\
\>\(\Rightarrow\) c\(^{0}\) = c\(^{1}\) \\
\\
If m\(_{1}\)  is encrypted: \\
\>Enc(m\(_{1}\)) = Enc(m\(_{1}^{0}\) \(\vert \vert\) m\(_{1}^{1}\)) = Enc(110 011) = f(m\(_{1}^{0}\)) \(\vert \vert\) f(m\(_{1}^{1}\)) =
f(110) \(\vert \vert\) f(011) = c\(^{0}\) \(\vert \vert\) c\(^{1}\) \\
\>\(\Rightarrow\) c\(^{0}\) \(\neq\) c\(^{1}\) \\
\\
If b = 0: c\(_{b}\) = c\(^{0}\) \(\vert \vert\) c\(^{1}\), c\(^{0}\) = c\(^{1}\) \\
If b = 1: c\(_{b}\) = c\(^{0}\) \(\vert \vert\) c\(^{1}\), c\(^{0}\) \(\neq\) c\(^{1}\) \\
This shows that even when the adversary can't break the encryption of the pseudorandom permutation the adversary can distinguish what message was encrypted with a probability of 100\(\%\) \\
\(\Rightarrow\) ECB mode is not EAU-secure \\
  