\pagebreak
a)\\
Let $F$ be a pseudorandom permutation. Then $F$ and $F^{-1}$ are pseudorandom permutations.\\
\\
$\mathrm{\Pi_{M} = (Gen, Mac, Vrfy)}$\\
\vskip 0.05in
\begin{tabular}{l@{\hskip 1in}l@{\hskip 0.5in}l}
$\underline{\mathrm{Gen(1^{\lambda })}}$ & $\underline{\mathrm{Mac_{k}(c)}}$ & $\underline{\mathrm{Vrfy_{k}(c,t)}}$ \\
$k \leftarrow \mathrm{Gen(1^{\lambda})}$ & $t \leftarrow \mathrm{F}_{k}^{-1}(c)$ & if $t = \mathrm{Mac}_{k}(m)$\\
\bf{return} $k$ & \bf{return} $t$ & \indent\bf{return} $1$\\
 & & \bf{return} $0$\\
\end{tabular}\\
\vskip 1cm
\noindent$\mathrm{\Pi_{E} = (Gen, Enc, Dec)}$\\
\vskip 0.05in
\begin{tabular}{l@{\hskip 1in}l@{\hskip 0.5in}l}
$\underline{\mathrm{Gen(1^{\lambda })}}$ & $\underline{\mathrm{Enc_{k}(m)}}$ & $\underline{\mathrm{Dec_{k}(c)}}$ \\
$k \leftarrow \mathrm{Gen(1^{\lambda})}$ & $r \leftarrow \{0,1\}^{\frac{n}{2}}$ & $v := \mathrm{F}_{k}^{-1}(c)$\\
\bf{return} $k$ & $m \in  \{0,1\}^{\frac{n}{2}}$ & \bf{return} $\mathrm{first}$ $\frac{n}{2}$ $\mathrm{bits}$ $\mathrm{of}$  $v$  \\
 & $c \leftarrow \mathrm{F}_{k}(m \parallel r)$ &  \\
 & \bf{return} $c$ & \\
\end{tabular}
\vskip 1cm
\noindent $\mathrm{Because}$ $\mathrm{Enc}_{k}(m), \mathrm{Mac}_{k}(\mathrm{Enc}_{k}(m)) = \mathrm{F}_{k}(m\parallel r), \mathrm{F}_{k}^{-1}(\mathrm{F}_{k}(m\parallel r)) = \mathrm{F}_{k}(m\parallel r), (m\parallel r)$\\
\vskip 1cm
\noindent TODO Beweise\\
\textbf{Proof that $\mathrm{\Pi_{M}}$ is secure}\\
We reduce the security of the Mac to the pseudorandomness of the function $F^{-1}$.\\
Therefore, we first assume that the construction is not secure and therefore there exists an adversary A that wins MacForge with non-negligible probability $\varepsilon(\lambda )$. We use this adversary A to build a distinguisher for the pseudorandomness of $F_{k}$.\\
With the help of the oracle $O_{D}$ of the pseudorandomness, D answers the oracle requests of A by computing $t := O_{D}(m)$. If the oracle answers with a pseudorandom function, the view of A is identical to $MacForge_{A,\Pi'}(\lambda)$. Thus we have \\
$Pr\left [ D^{F_{k}^{-1}(\cdot )}(1^{\lambda }) = 1 \right ] = Pr\left [ MacForge_{A,\Pi}(\lambda ) = 1 \right ] = \varepsilon $\\
where $k \leftarrow  \{0,1\}^{\lambda }$.\\
If the oracle answers with a random function, then we simulate the game for a different MAC-scheme $\Pi'$.
Let $\Pi' = (Gen', Mac', Vrfy')$ be a message authentication code which is the same as $\Pi$, except it uses a truly random function $f$ instead of the pseudorandom function $F_{k}$. 
It is easy to see that \\
$Pr\left [ MacForge_{A,\Pi'}(\lambda ) = 1 \right ] \leq 2^{-\lambda } $\\
This is the case because for any message m, the value t is uniformly distributed in $\{0,1\}_{*}$ from the point of view of A. The view of A is identical to $MacForge_{A,\Pi}(\lambda )$. We have \\
$Pr\left [ D^{f(\cdot )}(1^{\lambda }) = 1 \right ] = Pr\left [ MacForge_{A,\Pi'}(\lambda ) = 1 \right ]  \leq \frac{1}{2^{\lambda }} $ \\
where $f \leftarrow Func_{\lambda }$.\\
The distinguisher can now distinguish between pseudorandom and truly random with non-negligible probability.
As we assumed the function $F^{-1}$ to be pseudorandom, this is a contradiction and thus such an adversary cannot exist. Hence the MAC construction is secure.



