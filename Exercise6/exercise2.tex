\textbf{Task:} Show that \(\Pi_{CBC}\) is not CCA-secure by demonstrating a successful adversary.  \\
Assume n = 3 \\
The adversary  \(\mathcal{A}\) can choose the two messages \(m_{0}  = m_0^1 \vert \vert m_0^2 = 000\ 000\) and  \(m_{1}  = m_1^1 \vert \vert m_1^2 = 111\ 111\) which he sends to the challenger. Then he gets the ciphertext \(c_b = (c_b^0 \vert \vert c_b^1 \vert \vert c_b^2) = (IV \vert \vert f_k(IV \xor m_b^1) \vert \vert  f_k( f_k(IV \xor m_b^1) \xor m_b^2))\) back.\\
Then \(\mathcal{A}\) flipps the last bit from \(c_b^2\), so \((c_b^2)' = c_b^2 \xor 001\) and asks the decryption oracle for the decryption of \(c_b' = c_b^0 \vert \vert c_b^1 \vert \vert (c_b^2)'\). Because \(c_b' \neq c_b\) the decryption oracle answers with \(m' = f_k^{-1}(c_b^1) \xor c_b^0 \vert \vert f_k^{-1}(c_b^2) \xor c_b^1  = f_k^{-1}(f_k(IV \xor m_b^1)) \xor IV \vert \vert f_k^{-1}((c_b^2)') \xor f_k(IV \xor m_b^1) = m_b^1 \vert \vert f_k^{-1}((c_b^2)') \xor f_k(IV \xor m_b^1)\) \\ 
\(m_b^1\) is now either \(m_0^1\) or \(m_1^1\) because the change in \((c_b^2)'\) doesn't impact \(m_b^1\). So the adversary can say for sure, if the recieved civertext \(c_b\) is the encoding for \(m_0\) or \(m_1\). \\
\(\Rightarrow\) \(\Pi_{CBC}\) mode is not CCA-secure \\
  