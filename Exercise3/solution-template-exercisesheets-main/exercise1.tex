Trying some values:\\
\begin{tabular}{llllll}
	\(k = 2\) & \(a = 27\) & \(b = 72\) & \(c = \vert b - a \vert = 72 - 27 = 45 \) & \( d = 45\) & \(e = 54\)\\
	\(k = 3\) & \(a = 398\) & \(b = 983\) & \(c = \vert 983 - 398 \vert = 585 \) & \( d = 585\) & \(e = 558\)\\
	\(k = 3\) & \(a = 398\) & \(b = 938\) & \(c = \vert 983 - 938 \vert = 441\) & \( d = 441\) & \(e = 144\)\\
	\(k = 3\) & \(a = 321\) & \(b = 213\) & \(c = \vert 321 - 213 \vert = 108\) & \( d = 18\) & \(e = 81\)\\
\end{tabular}\\
\\
In the table above one can see that the sum of the digits of \(d\), respectivly \(e\) is always a multiple of \(9\).\\
The reason for this lies in the construction of the 'pseudorandom generator'.
\(b\) has the same digits like \(a\) only in an other order. To get \(c\) we subtract the greater number of those from the smaller, so \(c\) is always positive. Then we remove all 0-digits to get \(d\) and scramble the letters again for \(e\). The sum of digits didn't change after the computation of \(c\), so we look at \(c\) to argue that this sum is always a multiple of \(9\).\\
If \(a > b\) we compute for the sum of the digits of \(c = a - b\): \((a_1 - b_1) + (a_2 - b_2) + ... + (a_n - b_n) = a_1 + a_2 + ... + a_n - (b_1 + b_2 + ... + b_n)\). But this only hold if \(a_n > b_n\). \\
If \(a_i < b_i\) then it is \((a_1 - b_1) + (a_2 - b_2) +... + (a_{i-1} - b_{i-1} - 1) + (a_i - b_i + 10) + ... + (a_n - b_n) = a_1 + a_2 + ... + a_n - (b_1 + b_2 + ... + b_n) + 9\)\\ 
The sum of the digit of \(a\) has to be the same like the sum of the digits from \(b\), because \(b\) has the same digits like \(a\) only in an other order, so \(a_1 + a_2 + ... + a_n - (b_1 + b_2 + ... + b_n) = 0\).\\
If \(a_i < b_i\) holds for \(y\) positions the sum of the digits is \(9\cdot y \). \(a\) and \(b\) have the same digits, so \(a_i < b_i\) holds at least for one position, so \(y \ge 1\).\\
This argumentation is the same for \(b > a\), so the sum of the digits from \(c\), respectively \(d\) or \(e\), is always a multiple from \(9\).\\
If now all but the last digit of \(e\) are given one can always determine the last digit, because the sum of all digits has to be a multiple of \(9\). If the sum of the digits is already \(9\), the last bit has to be \(9\) as well, because all 0-digits had erased. If this is not the case, we could not determine if the last digit has to be \(0\) or \(9\).\\
So the described generator does not pass the next-character test.
