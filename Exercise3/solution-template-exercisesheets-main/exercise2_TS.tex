% % % % \input{exercise2_TS.tex} &In exercise2.tex Datei einfügen

\begin{itemize}
\item[(a)]
	Is G\(_{a}\)(s) = G(s)\(\vert \vert\)0 a secure PRG? \\
	No since the last bit is always 1. This bit is not uniformly at random because the probablitiy of that bit being 1 is 100\(\%\) instead of 50\(\%\). \\
	\(\Rightarrow\)  G\(_{a}\) is a not a PRG
\item[(b)]
	Is G\(_{b}\)(s\(\vert \vert\)b) = G(s)\(\vert \vert\)b where \(\vert\)b\(\vert\) = 1 a secure PRG? \\
	Yes because adding a single random but fixed bit has the probability 50\(\%\) being 1 and  50\(\%\) being 0 meaning b is uniformly random.
	b = \(\{0,1\}\) is a pseudorandom generator and the concatination of two pseudorandom number generators is a pseudorandom number generator itself.
	Also since b is part of the argument G\(_{b}\) is deterministic. \\
	\(\Rightarrow\)  G\(_{b}\) is a secure PRG
\item[(c)]
	Is G\(_{c}\)(s) = G(s\(\vert \vert\)0) a secure PRG? \\
	
\item[(d)]
	Is G\(_{d}\)(s) = G(s\(\vert \vert\)\(0^{\vert s \vert}\)) a secure PRG? \\

\item[(e)]
	Is G\(_{e}\)(s) = G(s) \(\xor\) \(1^{ l(\vert s \vert}\) a secure PRG? \\
	Yes because G(s)  \(\xor\) \(1^{ l(\vert s \vert}\) = G(s) and G(s) is per definition a secure PRG. \\
	\(\Rightarrow\) Therefor G\(_{e}\)(s) is also a secure PRG.
\item[(f)]
	Is G\(_{f}\)(s) = \textit{trunc}(G(\textit{trunc}(s))) a secure PRG? \\
	where \textit{trunc}(x) for a nonempty string x denotes all but the last bit of x. \\
	(For this part, assume that l(n) >  n + 2, and ignore the fact that G\(_{f}\) is undefined on input strings of length 1.) \\
	\\
	
\end{itemize}
 &In exercise2.tex Datei einfügen

\begin{itemize}
\item[(a)]
	Is G\(_{a}\)(s) = G(s)\(\vert \vert\)0 a secure PRG? \\
	No since the last bit is always 1. This bit is not uniformly at random because the probablitiy of that bit being 1 is 100\(\%\) instead of 50\(\%\). \\
	\(\Rightarrow\)  G\(_{a}\) is a not a PRG
\item[(b)]
	Is G\(_{b}\)(s\(\vert \vert\)b) = G(s)\(\vert \vert\)b where \(\vert\)b\(\vert\) = 1 a secure PRG? \\
	Yes because adding a single random but fixed bit has the probability 50\(\%\) being 1 and  50\(\%\) being 0 meaning b is uniformly random.
	b = \(\{0,1\}\) is a pseudorandom generator and the concatination of two pseudorandom number generators is a pseudorandom number generator itself.
	Also since b is part of the argument G\(_{b}\) is deterministic. \\
	\(\Rightarrow\)  G\(_{b}\) is a secure PRG
\item[(c)]
	Is G\(_{c}\)(s) = G(s\(\vert \vert\)0) a secure PRG? \\
	
\item[(d)]
	Is G\(_{d}\)(s) = G(s\(\vert \vert\)\(0^{\vert s \vert}\)) a secure PRG? \\

\item[(e)]
	Is G\(_{e}\)(s) = G(s) \(\xor\) \(1^{ l(\vert s \vert}\) a secure PRG? \\
	Yes because G(s)  \(\xor\) \(1^{ l(\vert s \vert}\) = G(s) and G(s) is per definition a secure PRG. \\
	\(\Rightarrow\) Therefor G\(_{e}\)(s) is also a secure PRG.
\item[(f)]
	Is G\(_{f}\)(s) = \textit{trunc}(G(\textit{trunc}(s))) a secure PRG? \\
	where \textit{trunc}(x) for a nonempty string x denotes all but the last bit of x. \\
	(For this part, assume that l(n) >  n + 2, and ignore the fact that G\(_{f}\) is undefined on input strings of length 1.) \\
	\\
	
\end{itemize}
 &In exercise2.tex Datei einfügen

\begin{itemize}
\item[(a)]
	Is G\(_{a}\)(s) = G(s)\(\vert \vert\)0 a secure PRG? \\
	No since the last bit is always 1. This bit is not uniformly at random because the probablitiy of that bit being 1 is 100\(\%\) instead of 50\(\%\). \\
	\(\Rightarrow\)  G\(_{a}\) is a not a PRG
\item[(b)]
	Is G\(_{b}\)(s\(\vert \vert\)b) = G(s)\(\vert \vert\)b where \(\vert\)b\(\vert\) = 1 a secure PRG? \\
	Yes because adding a single random but fixed bit has the probability 50\(\%\) being 1 and  50\(\%\) being 0 meaning b is uniformly random.
	b = \(\{0,1\}\) is a pseudorandom generator and the concatination of two pseudorandom number generators is a pseudorandom number generator itself.
	Also since b is part of the argument G\(_{b}\) is deterministic. \\
	\(\Rightarrow\)  G\(_{b}\) is a secure PRG
\item[(c)]
	Is G\(_{c}\)(s) = G(s\(\vert \vert\)0) a secure PRG? \\
	
\item[(d)]
	Is G\(_{d}\)(s) = G(s\(\vert \vert\)\(0^{\vert s \vert}\)) a secure PRG? \\

\item[(e)]
	Is G\(_{e}\)(s) = G(s) \(\xor\) \(1^{ l(\vert s \vert}\) a secure PRG? \\
	Yes because G(s)  \(\xor\) \(1^{ l(\vert s \vert}\) = G(s) and G(s) is per definition a secure PRG. \\
	\(\Rightarrow\) Therefor G\(_{e}\)(s) is also a secure PRG.
\item[(f)]
	Is G\(_{f}\)(s) = \textit{trunc}(G(\textit{trunc}(s))) a secure PRG? \\
	where \textit{trunc}(x) for a nonempty string x denotes all but the last bit of x. \\
	(For this part, assume that l(n) >  n + 2, and ignore the fact that G\(_{f}\) is undefined on input strings of length 1.) \\
	\\
	
\end{itemize}
 &In exercise2.tex Datei einfügen

\begin{itemize}
\item[(a)]
	Is G\(_{a}\)(s) = G(s)\(\vert \vert\)0 a secure PRG? \\
	No since the last bit is always 1. This bit is not uniformly at random because the probablitiy of that bit being 1 is 100\(\%\) instead of 50\(\%\). \\
	\(\Rightarrow\)  G\(_{a}\) is a not a PRG
\item[(b)]
	Is G\(_{b}\)(s\(\vert \vert\)b) = G(s)\(\vert \vert\)b where \(\vert\)b\(\vert\) = 1 a secure PRG? \\
	Yes because adding a single random but fixed bit has the probability 50\(\%\) being 1 and  50\(\%\) being 0 meaning b is uniformly random.
	b = \(\{0,1\}\) is a pseudorandom generator and the concatination of two pseudorandom number generators is a pseudorandom number generator itself.
	Also since b is part of the argument G\(_{b}\) is deterministic. \\
	\(\Rightarrow\)  G\(_{b}\) is a secure PRG
\item[(c)]
	Is G\(_{c}\)(s) = G(s\(\vert \vert\)0) a secure PRG? \\
	
\item[(d)]
	Is G\(_{d}\)(s) = G(s\(\vert \vert\)\(0^{\vert s \vert}\)) a secure PRG? \\

\item[(e)]
	Is G\(_{e}\)(s) = G(s) \(\xor\) \(1^{ l(\vert s \vert}\) a secure PRG? \\
	Yes because G(s)  \(\xor\) \(1^{ l(\vert s \vert}\) = G(s) and G(s) is per definition a secure PRG. \\
	\(\Rightarrow\) Therefor G\(_{e}\)(s) is also a secure PRG.
\item[(f)]
	Is G\(_{f}\)(s) = \textit{trunc}(G(\textit{trunc}(s))) a secure PRG? \\
	where \textit{trunc}(x) for a nonempty string x denotes all but the last bit of x. \\
	(For this part, assume that l(n) >  n + 2, and ignore the fact that G\(_{f}\) is undefined on input strings of length 1.) \\
	\\
	
\end{itemize}
